% Options for packages loaded elsewhere
\PassOptionsToPackage{unicode}{hyperref}
\PassOptionsToPackage{hyphens}{url}
%
\documentclass[
]{article}
\usepackage{amsmath,amssymb}
\usepackage{lmodern}
\usepackage{iftex}
\ifPDFTeX
  \usepackage[T1]{fontenc}
  \usepackage[utf8]{inputenc}
  \usepackage{textcomp} % provide euro and other symbols
\else % if luatex or xetex
  \usepackage{unicode-math}
  \defaultfontfeatures{Scale=MatchLowercase}
  \defaultfontfeatures[\rmfamily]{Ligatures=TeX,Scale=1}
\fi
% Use upquote if available, for straight quotes in verbatim environments
\IfFileExists{upquote.sty}{\usepackage{upquote}}{}
\IfFileExists{microtype.sty}{% use microtype if available
  \usepackage[]{microtype}
  \UseMicrotypeSet[protrusion]{basicmath} % disable protrusion for tt fonts
}{}
\makeatletter
\@ifundefined{KOMAClassName}{% if non-KOMA class
  \IfFileExists{parskip.sty}{%
    \usepackage{parskip}
  }{% else
    \setlength{\parindent}{0pt}
    \setlength{\parskip}{6pt plus 2pt minus 1pt}}
}{% if KOMA class
  \KOMAoptions{parskip=half}}
\makeatother
\usepackage{xcolor}
\usepackage[margin=1in]{geometry}
\usepackage{color}
\usepackage{fancyvrb}
\newcommand{\VerbBar}{|}
\newcommand{\VERB}{\Verb[commandchars=\\\{\}]}
\DefineVerbatimEnvironment{Highlighting}{Verbatim}{commandchars=\\\{\}}
% Add ',fontsize=\small' for more characters per line
\usepackage{framed}
\definecolor{shadecolor}{RGB}{248,248,248}
\newenvironment{Shaded}{\begin{snugshade}}{\end{snugshade}}
\newcommand{\AlertTok}[1]{\textcolor[rgb]{0.94,0.16,0.16}{#1}}
\newcommand{\AnnotationTok}[1]{\textcolor[rgb]{0.56,0.35,0.01}{\textbf{\textit{#1}}}}
\newcommand{\AttributeTok}[1]{\textcolor[rgb]{0.77,0.63,0.00}{#1}}
\newcommand{\BaseNTok}[1]{\textcolor[rgb]{0.00,0.00,0.81}{#1}}
\newcommand{\BuiltInTok}[1]{#1}
\newcommand{\CharTok}[1]{\textcolor[rgb]{0.31,0.60,0.02}{#1}}
\newcommand{\CommentTok}[1]{\textcolor[rgb]{0.56,0.35,0.01}{\textit{#1}}}
\newcommand{\CommentVarTok}[1]{\textcolor[rgb]{0.56,0.35,0.01}{\textbf{\textit{#1}}}}
\newcommand{\ConstantTok}[1]{\textcolor[rgb]{0.00,0.00,0.00}{#1}}
\newcommand{\ControlFlowTok}[1]{\textcolor[rgb]{0.13,0.29,0.53}{\textbf{#1}}}
\newcommand{\DataTypeTok}[1]{\textcolor[rgb]{0.13,0.29,0.53}{#1}}
\newcommand{\DecValTok}[1]{\textcolor[rgb]{0.00,0.00,0.81}{#1}}
\newcommand{\DocumentationTok}[1]{\textcolor[rgb]{0.56,0.35,0.01}{\textbf{\textit{#1}}}}
\newcommand{\ErrorTok}[1]{\textcolor[rgb]{0.64,0.00,0.00}{\textbf{#1}}}
\newcommand{\ExtensionTok}[1]{#1}
\newcommand{\FloatTok}[1]{\textcolor[rgb]{0.00,0.00,0.81}{#1}}
\newcommand{\FunctionTok}[1]{\textcolor[rgb]{0.00,0.00,0.00}{#1}}
\newcommand{\ImportTok}[1]{#1}
\newcommand{\InformationTok}[1]{\textcolor[rgb]{0.56,0.35,0.01}{\textbf{\textit{#1}}}}
\newcommand{\KeywordTok}[1]{\textcolor[rgb]{0.13,0.29,0.53}{\textbf{#1}}}
\newcommand{\NormalTok}[1]{#1}
\newcommand{\OperatorTok}[1]{\textcolor[rgb]{0.81,0.36,0.00}{\textbf{#1}}}
\newcommand{\OtherTok}[1]{\textcolor[rgb]{0.56,0.35,0.01}{#1}}
\newcommand{\PreprocessorTok}[1]{\textcolor[rgb]{0.56,0.35,0.01}{\textit{#1}}}
\newcommand{\RegionMarkerTok}[1]{#1}
\newcommand{\SpecialCharTok}[1]{\textcolor[rgb]{0.00,0.00,0.00}{#1}}
\newcommand{\SpecialStringTok}[1]{\textcolor[rgb]{0.31,0.60,0.02}{#1}}
\newcommand{\StringTok}[1]{\textcolor[rgb]{0.31,0.60,0.02}{#1}}
\newcommand{\VariableTok}[1]{\textcolor[rgb]{0.00,0.00,0.00}{#1}}
\newcommand{\VerbatimStringTok}[1]{\textcolor[rgb]{0.31,0.60,0.02}{#1}}
\newcommand{\WarningTok}[1]{\textcolor[rgb]{0.56,0.35,0.01}{\textbf{\textit{#1}}}}
\usepackage{graphicx}
\makeatletter
\def\maxwidth{\ifdim\Gin@nat@width>\linewidth\linewidth\else\Gin@nat@width\fi}
\def\maxheight{\ifdim\Gin@nat@height>\textheight\textheight\else\Gin@nat@height\fi}
\makeatother
% Scale images if necessary, so that they will not overflow the page
% margins by default, and it is still possible to overwrite the defaults
% using explicit options in \includegraphics[width, height, ...]{}
\setkeys{Gin}{width=\maxwidth,height=\maxheight,keepaspectratio}
% Set default figure placement to htbp
\makeatletter
\def\fps@figure{htbp}
\makeatother
\setlength{\emergencystretch}{3em} % prevent overfull lines
\providecommand{\tightlist}{%
  \setlength{\itemsep}{0pt}\setlength{\parskip}{0pt}}
\setcounter{secnumdepth}{-\maxdimen} % remove section numbering
\ifLuaTeX
  \usepackage{selnolig}  % disable illegal ligatures
\fi
\IfFileExists{bookmark.sty}{\usepackage{bookmark}}{\usepackage{hyperref}}
\IfFileExists{xurl.sty}{\usepackage{xurl}}{} % add URL line breaks if available
\urlstyle{same} % disable monospaced font for URLs
\hypersetup{
  pdftitle={BayesIngenuo},
  pdfauthor={Cristopher Barrios, Carlos Daniel Estrada},
  hidelinks,
  pdfcreator={LaTeX via pandoc}}

\title{BayesIngenuo}
\author{Cristopher Barrios, Carlos Daniel Estrada}
\date{2023-03-17}

\begin{document}
\maketitle

librerias

\begin{Shaded}
\begin{Highlighting}[]
\FunctionTok{library}\NormalTok{(rpart)}
\FunctionTok{library}\NormalTok{(rpart.plot)}
\FunctionTok{library}\NormalTok{(dplyr) }
\end{Highlighting}
\end{Shaded}

\begin{verbatim}
## 
## Attaching package: 'dplyr'
\end{verbatim}

\begin{verbatim}
## The following objects are masked from 'package:stats':
## 
##     filter, lag
\end{verbatim}

\begin{verbatim}
## The following objects are masked from 'package:base':
## 
##     intersect, setdiff, setequal, union
\end{verbatim}

\begin{Shaded}
\begin{Highlighting}[]
\FunctionTok{library}\NormalTok{(fpc) }
\FunctionTok{library}\NormalTok{(cluster) }
\FunctionTok{library}\NormalTok{(}\StringTok{"ggpubr"}\NormalTok{) }
\end{Highlighting}
\end{Shaded}

\begin{verbatim}
## Loading required package: ggplot2
\end{verbatim}

\begin{Shaded}
\begin{Highlighting}[]
\FunctionTok{library}\NormalTok{(mclust)}
\end{Highlighting}
\end{Shaded}

\begin{verbatim}
## Package 'mclust' version 6.0.0
## Type 'citation("mclust")' for citing this R package in publications.
\end{verbatim}

\begin{Shaded}
\begin{Highlighting}[]
\FunctionTok{library}\NormalTok{(caret)}
\end{Highlighting}
\end{Shaded}

\begin{verbatim}
## Loading required package: lattice
\end{verbatim}

\begin{Shaded}
\begin{Highlighting}[]
\FunctionTok{library}\NormalTok{(tree)}
\FunctionTok{library}\NormalTok{(randomForest)}
\end{Highlighting}
\end{Shaded}

\begin{verbatim}
## randomForest 4.7-1.1
\end{verbatim}

\begin{verbatim}
## Type rfNews() to see new features/changes/bug fixes.
\end{verbatim}

\begin{verbatim}
## 
## Attaching package: 'randomForest'
\end{verbatim}

\begin{verbatim}
## The following object is masked from 'package:ggplot2':
## 
##     margin
\end{verbatim}

\begin{verbatim}
## The following object is masked from 'package:dplyr':
## 
##     combine
\end{verbatim}

\begin{Shaded}
\begin{Highlighting}[]
\FunctionTok{library}\NormalTok{(plyr)}
\end{Highlighting}
\end{Shaded}

\begin{verbatim}
## ------------------------------------------------------------------------------
\end{verbatim}

\begin{verbatim}
## You have loaded plyr after dplyr - this is likely to cause problems.
## If you need functions from both plyr and dplyr, please load plyr first, then dplyr:
## library(plyr); library(dplyr)
\end{verbatim}

\begin{verbatim}
## ------------------------------------------------------------------------------
\end{verbatim}

\begin{verbatim}
## 
## Attaching package: 'plyr'
\end{verbatim}

\begin{verbatim}
## The following object is masked from 'package:ggpubr':
## 
##     mutate
\end{verbatim}

\begin{verbatim}
## The following objects are masked from 'package:dplyr':
## 
##     arrange, count, desc, failwith, id, mutate, rename, summarise,
##     summarize
\end{verbatim}

\begin{Shaded}
\begin{Highlighting}[]
\FunctionTok{library}\NormalTok{(}\StringTok{"stats"}\NormalTok{)}
\FunctionTok{library}\NormalTok{(}\StringTok{"datasets"}\NormalTok{)}
\FunctionTok{library}\NormalTok{(}\StringTok{"prediction"}\NormalTok{)}
\FunctionTok{library}\NormalTok{(tidyverse)}
\end{Highlighting}
\end{Shaded}

\begin{verbatim}
## -- Attaching core tidyverse packages ------------------------ tidyverse 2.0.0 --
## v forcats   1.0.0     v stringr   1.5.0
## v lubridate 1.9.2     v tibble    3.1.8
## v purrr     1.0.1     v tidyr     1.3.0
## v readr     2.1.4
\end{verbatim}

\begin{verbatim}
## -- Conflicts ------------------------------------------ tidyverse_conflicts() --
## x plyr::arrange()         masks dplyr::arrange()
## x randomForest::combine() masks dplyr::combine()
## x purrr::compact()        masks plyr::compact()
## x plyr::count()           masks dplyr::count()
## x plyr::desc()            masks dplyr::desc()
## x plyr::failwith()        masks dplyr::failwith()
## x dplyr::filter()         masks stats::filter()
## x plyr::id()              masks dplyr::id()
## x dplyr::lag()            masks stats::lag()
## x purrr::lift()           masks caret::lift()
## x purrr::map()            masks mclust::map()
## x randomForest::margin()  masks ggplot2::margin()
## x plyr::mutate()          masks ggpubr::mutate(), dplyr::mutate()
## x plyr::rename()          masks dplyr::rename()
## x plyr::summarise()       masks dplyr::summarise()
## x plyr::summarize()       masks dplyr::summarize()
## i Use the ]8;;http://conflicted.r-lib.org/conflicted package]8;; to force all conflicts to become errors
\end{verbatim}

\hypertarget{use-los-mismos-conjuntos-de-entrenamiento-y-prueba-que-utilizuxf3-en-las-dos-hojas-anteriores.}{%
\subsubsection{1. Use los mismos conjuntos de entrenamiento y prueba que
utilizó en las dos hojas
anteriores.}\label{use-los-mismos-conjuntos-de-entrenamiento-y-prueba-que-utilizuxf3-en-las-dos-hojas-anteriores.}}

\begin{Shaded}
\begin{Highlighting}[]
\NormalTok{datos }\OtherTok{=} \FunctionTok{read.csv}\NormalTok{(}\StringTok{"./train.csv"}\NormalTok{)}
\NormalTok{test}\OtherTok{\textless{}{-}} \FunctionTok{read.csv}\NormalTok{(}\StringTok{"./test.csv"}\NormalTok{, }\AttributeTok{stringsAsFactors =} \ConstantTok{FALSE}\NormalTok{)}
\end{Highlighting}
\end{Shaded}

\begin{Shaded}
\begin{Highlighting}[]
\NormalTok{set\_entrenamiento }\OtherTok{\textless{}{-}} \FunctionTok{sample\_frac}\NormalTok{(datos, .}\DecValTok{7}\NormalTok{)}
\NormalTok{set\_prueba }\OtherTok{\textless{}{-}}\FunctionTok{setdiff}\NormalTok{(datos, set\_entrenamiento)}


\NormalTok{drop }\OtherTok{\textless{}{-}} \FunctionTok{c}\NormalTok{(}\StringTok{"LotFrontage"}\NormalTok{, }\StringTok{"Alley"}\NormalTok{, }\StringTok{"MasVnrType"}\NormalTok{, }\StringTok{"MasVnrArea"}\NormalTok{, }\StringTok{"BsmtQual"}\NormalTok{, }\StringTok{"BsmtCond"}\NormalTok{, }\StringTok{"BsmtExposure"}\NormalTok{, }\StringTok{"BsmtFinType1"}\NormalTok{, }\StringTok{"BsmtFinType2"}\NormalTok{, }\StringTok{"Electrical"}\NormalTok{, }\StringTok{"FireplaceQu"}\NormalTok{, }\StringTok{"GarageType"}\NormalTok{, }\StringTok{"GarageYrBlt"}\NormalTok{, }\StringTok{"GarageFinish"}\NormalTok{, }\StringTok{"GarageQual"}\NormalTok{, }\StringTok{"GarageCond"}\NormalTok{, }\StringTok{"PoolQC"}\NormalTok{, }\StringTok{"Fence"}\NormalTok{, }\StringTok{"MiscFeature"}\NormalTok{)}
\NormalTok{set\_entrenamiento }\OtherTok{\textless{}{-}}\NormalTok{ set\_entrenamiento[, }\SpecialCharTok{!}\NormalTok{(}\FunctionTok{names}\NormalTok{(set\_entrenamiento) }\SpecialCharTok{\%in\%}\NormalTok{ drop)]}
\NormalTok{set\_prueba }\OtherTok{\textless{}{-}}\NormalTok{ set\_prueba[, }\SpecialCharTok{!}\NormalTok{(}\FunctionTok{names}\NormalTok{(set\_prueba) }\SpecialCharTok{\%in\%}\NormalTok{ drop)]}
\end{Highlighting}
\end{Shaded}

\hypertarget{elabore-un-modelo-de-regresiuxf3n-usando-bayes-ingenuo-naive-bayes-el-conjunto-de-entrenamiento-y-la-variable-respuesta-salesprice.-prediga-con-el-modelo-y-explique-los-resultados-a-los-que-llega.-aseguxfarese-que-los-conjuntos-de-entrenamiento-y-prueba-sean-los-mismos-de-las-hojas-anteriores-para-que-los-modelos-sean-comparables.}{%
\subsubsection{2. Elabore un modelo de regresión usando bayes ingenuo
(naive bayes), el conjunto de entrenamiento y la variable respuesta
SalesPrice. Prediga con el modelo y explique los resultados a los que
llega. Asegúrese que los conjuntos de entrenamiento y prueba sean los
mismos de las hojas anteriores para que los modelos sean
comparables.}\label{elabore-un-modelo-de-regresiuxf3n-usando-bayes-ingenuo-naive-bayes-el-conjunto-de-entrenamiento-y-la-variable-respuesta-salesprice.-prediga-con-el-modelo-y-explique-los-resultados-a-los-que-llega.-aseguxfarese-que-los-conjuntos-de-entrenamiento-y-prueba-sean-los-mismos-de-las-hojas-anteriores-para-que-los-modelos-sean-comparables.}}

\hypertarget{haga-un-modelo-de-clasificaciuxf3n-use-la-variable-categuxf3rica-que-hizo-con-el-precio-de-las-casas-barata-media-y-cara-como-variable-respuesta.}{%
\subsubsection{3. Haga un modelo de clasificación, use la variable
categórica que hizo con el precio de las casas (barata, media y cara)
como variable
respuesta.}\label{haga-un-modelo-de-clasificaciuxf3n-use-la-variable-categuxf3rica-que-hizo-con-el-precio-de-las-casas-barata-media-y-cara-como-variable-respuesta.}}

\hypertarget{utilice-los-modelos-con-el-conjunto-de-prueba-y-determine-la-eficiencia-del-algoritmo-para-predecir-y-clasificar.}{%
\subsubsection{4. Utilice los modelos con el conjunto de prueba y
determine la eficiencia del algoritmo para predecir y
clasificar.}\label{utilice-los-modelos-con-el-conjunto-de-prueba-y-determine-la-eficiencia-del-algoritmo-para-predecir-y-clasificar.}}

\hypertarget{analice-los-resultados-del-modelo-de-regresiuxf3n.-quuxe9-tan-bien-le-fue-prediciendo}{%
\subsubsection{5. Analice los resultados del modelo de regresión. ¿Qué
tan bien le fue
prediciendo?}\label{analice-los-resultados-del-modelo-de-regresiuxf3n.-quuxe9-tan-bien-le-fue-prediciendo}}

\hypertarget{compare-los-resultados-con-el-modelo-de-regresiuxf3n-lineal-y-el-uxe1rbol-de-regresiuxf3n-que-hizo-en-las-hojas-pasadas.-cuuxe1l-funcionuxf3-mejor}{%
\subsubsection{6. Compare los resultados con el modelo de regresión
lineal y el árbol de regresión que hizo en las hojas pasadas. ¿Cuál
funcionó
mejor?}\label{compare-los-resultados-con-el-modelo-de-regresiuxf3n-lineal-y-el-uxe1rbol-de-regresiuxf3n-que-hizo-en-las-hojas-pasadas.-cuuxe1l-funcionuxf3-mejor}}

\hypertarget{haga-un-anuxe1lisis-de-la-eficiencia-del-modelo-de-clasificaciuxf3n-usando-una-matriz-de-confusiuxf3n.-tenga-en-cuenta-la-efectividad-donde-el-algoritmo-se-equivocuxf3-muxe1s-donde-se-equivocuxf3-menos-y-la-importancia-que-tienen-los-errores.}{%
\subsubsection{7. Haga un análisis de la eficiencia del modelo de
clasificación usando una matriz de confusión. Tenga en cuenta la
efectividad, donde el algoritmo se equivocó más, donde se equivocó menos
y la importancia que tienen los
errores.}\label{haga-un-anuxe1lisis-de-la-eficiencia-del-modelo-de-clasificaciuxf3n-usando-una-matriz-de-confusiuxf3n.-tenga-en-cuenta-la-efectividad-donde-el-algoritmo-se-equivocuxf3-muxe1s-donde-se-equivocuxf3-menos-y-la-importancia-que-tienen-los-errores.}}

\hypertarget{analice-el-modelo.-cree-que-pueda-estar-sobre-ajustado}{%
\subsubsection{8. Analice el modelo. ¿Cree que pueda estar sobre
ajustado?}\label{analice-el-modelo.-cree-que-pueda-estar-sobre-ajustado}}

\hypertarget{haga-un-modelo-usando-validaciuxf3n-cruzada-compare-los-resultados-de-este-con-los-del-modelo-anterior.-cuuxe1l-funcionuxf3-mejor}{%
\subsubsection{9. Haga un modelo usando validación cruzada, compare los
resultados de este con los del modelo anterior. ¿Cuál funcionó
mejor?}\label{haga-un-modelo-usando-validaciuxf3n-cruzada-compare-los-resultados-de-este-con-los-del-modelo-anterior.-cuuxe1l-funcionuxf3-mejor}}

\hypertarget{compare-la-eficiencia-del-algoritmo-con-el-resultado-obtenido-con-el-uxe1rbol-de-decisiuxf3n-el-de-clasificaciuxf3n-y-el-modelo-de-random-forest-que-hizo-en-la-hoja-pasada.-cuuxe1l-es-mejor-para-predecir-cuuxe1l-se-demoruxf3-muxe1s-en-procesar}{%
\subsubsection{10. Compare la eficiencia del algoritmo con el resultado
obtenido con el árbol de decisión (el de clasificación) y el modelo de
random forest que hizo en la hoja pasada. ¿Cuál es mejor para predecir?
¿Cuál se demoró más en
procesar?}\label{compare-la-eficiencia-del-algoritmo-con-el-resultado-obtenido-con-el-uxe1rbol-de-decisiuxf3n-el-de-clasificaciuxf3n-y-el-modelo-de-random-forest-que-hizo-en-la-hoja-pasada.-cuuxe1l-es-mejor-para-predecir-cuuxe1l-se-demoruxf3-muxe1s-en-procesar}}

\end{document}
